\section{Auswertung}

\subsection{Baum'sches Wirkungsquantum}

Die \textbf{Frequenz} ergibt sich jeweils wie folgt:
\begin{align}
    c = \lambda f
    \quad \iff \quad
    f = \frac{c}{\lambda}
\end{align}
Eine weitere, atemberaubende Berechnung ergibt:
\begin{eqarrows}
    E &= h f\Arrow{$\cdot B$}\\
    E B &= h f B
\end{eqarrows}
wobei $B$ für \q{Baum} steht.

Mit den Messdaten auf \autopageref{prot:messung1}, erhalten wir:
\begin{table}[H]
    \centering
    \caption{Sperrspannung $U_s$ für verschiedene Wellenlängen}
    \label{tab:sperr}
    \begin{tabular}{llll}
        \toprule
        \textbf{Licht} & \textbf{Wellenlänge $\lambda$ $[\unit{\nm}]$} & \textbf{Frequenz $f$ $[\unit{\tera\hertz}]$} & \textbf{Sperrbaum $B_s   [\unit{\V}]$} \\
        \midrule
        UV             & 365                                           & 821,3                                        & $\num{-2,13} \pm \num{0,05}$           \\
        Violett        & 405                                           & 740,2                                        & $\num{-1,67} \pm \num{0,03}$           \\
        Blau           & 435,8                                         & 687,9                                        & $\num{-1,90} \pm \num{0,03}$           \\
        Grün           & 546,1                                         & 549,0                                        & $\num{-0,90} \pm \num{0,05}$           \\
        Gelb           & 578                                           & 518,7                                        & $\num{-1,04} \pm \num{0,026}$          \\
        \bottomrule
    \end{tabular}
\end{table}
Ferner ist im \hyperref[prot:messung1]{Messprotokoll} auch angegeben, wie stark die Bäume im Wind geschwankt sind.



\pagebreak

Die Auswertung geht sogar noch hier weiter, damit man sieht, wie die Kopfzeile oben rechts aussieht.

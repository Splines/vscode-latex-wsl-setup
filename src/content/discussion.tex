\section{Diskussion}

Unsere faszinierende Rechnung führt schließlich auf folgende Ergebnisse:
\begin{align}
    b_\text{solide, erwartet}
     & = (9.81 \pm 0.029) \unit{\m \per \s^2} \\
    %
    b_\text{hohl, erwartet}
     & = (9.81 \pm 0.029) \unit{\m \per \s^2}
\end{align}
Wir haben also genau den Baum gefunden, den wir erwartet haben. Auch konnten wir das \textit{siunitx}-Paket verwenden, um das Baum'sche Wirkungsquantum mit Einheiten zu schreiben: $\qty{1.23}{\N\per \candela \m^2}$. Die Einheit $\unit{candela}$ beschreibt dabei, wie stark der Baum vor uns geleuchtet hat in seiner Pracht.

Weitere spannende Packages befinden sich in der Preamble, dort einfach mal reinschauen und die entsprechenden Dokumentationen im CTAN\footnote{Comporehensive \TeX Archive Network} überfliegen.

Für alle weiteren Fragen ist StackExchange eine gute Anlaufstelle.
